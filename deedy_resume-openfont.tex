%%%%%%%%%%%%%%%%%%%%%%%%%%%%%%%%%%%%%%%
% Deedy - One Page Two Column Resume
% LaTeX Template
% Version 1.2 (16/9/2014)
%
%
% IMPORTANT: THIS TEMPLATE NEEDS TO BE COMPILED WITH XeLaTeX
%
% This template uses several fonts not included with Windows/Linux by
% default. If you get compilation errors saying a font is missing, find the line
% on which the font is used and either change it to a font included with your
% operating system or comment the line out to use the default font.
% 
%%%%%%%%%%%%%%%%%%%%%%%%%%%%%%%%%%%%%%
% 
% TODO:
% 1. Integrate biber/bibtex for article citation under publications.
% 2. Figure out a smoother way for the document to flow onto the next page.
% 3. Add styling information for a "Projects/Hacks" section.
% 4. Add location/address information
% 5. Merge OpenFont and MacFonts as a single sty with options.
% 
%%%%%%%%%%%%%%%%%%%%%%%%%%%%%%%%%%%%%%
%
% CHANGELOG:
% v1.1:
% 1. Fixed several compilation bugs with \renewcommand
% 2. Got Open-source fonts (Windows/Linux support)
% 3. Added Last Updated
% 4. Move Title styling into .sty
% 5. Commented .sty file.
%
%%%%%%%%%%%%%%%%%%%%%%%%%%%%%%%%%%%%%%%
%
% Known Issues:
% 1. Overflows onto second page if any column's contents are more than the
% vertical limit
% 2. Hacky space on the first bullet point on the second column.
%
%%%%%%%%%%%%%%%%%%%%%%%%%%%%%%%%%%%%%%


\documentclass[]{deedy-resume-openfont}
\usepackage{fancyhdr}
 
\pagestyle{fancy}
\fancyhf{}
 
\begin{document}

%%%%%%%%%%%%%%%%%%%%%%%%%%%%%%%%%%%%%%
%
%     
%
%%%%%%%%%%%%%%%%%%%%%%%%%%%%%%%%%%%%%%
\lastupdated

%%%%%%%%%%%%%%%%%%%%%%%%%%%%%%%%%%%%%%
%
%     TITLE NAME
%
%%%%%%%%%%%%%%%%%%%%%%%%%%%%%%%%%%%%%%
\namesection{Tarannum}{Perween}{ \urlstyle{same}\href{https://tarannum-perween.github.io/}{https://tarannum-perween.github.io/} \\
\href{mailto:tperween20@gmail.com}{tperween20@gmail.com} | 6200015660 | \href{mailto:183103@nith.ac.in}{183103@nith.ac.in}
}

%%%%%%%%%%%%%%%%%%%%%%%%%%%%%%%%%%%%%%
%
%     COLUMN ONE
%
%%%%%%%%%%%%%%%%%%%%%%%%%%%%%%%%%%%%%%

\begin{minipage}[t]{0.33\textwidth} 

%%%%%%%%%%%%%%%%%%%%%%%%%%%%%%%%%%%%%%
%     EDUCATION
%%%%%%%%%%%%%%%%%%%%%%%%%%%%%%%%%%%%%%

\section{Education} 

\subsection{NIT Hamirpur}
\descript{B.Tech in Mechanical Engineering}
\location{2018-2022 |  Hamirpur, H.P}
\location{ CGPI: 8.15 / 10.0}
\sectionsep

\subsection{S.N. SAHAY College}
\descript{Intermediate}
\location{2015-2017 |  Muzaffarpur, Bihar}
\sectionsep

\subsection{St Joseph’s senior\\ secondary school}
\descript{Matriculation}
\location{May 2015 |  Muzaffarpur, Bihar}
\sectionsep




%%%%%%%%%%%%%%%%%%%%%%%%%%%%%%%%%%%%%%
%     LINKS
%%%%%%%%%%%%%%%%%%%%%%%%%%%%%%%%%%%%%%

\section{Links} 
Github:// \href{https://github.com/tarannum-perween}{\bf tarannum-perween} \\
LinkedIn://  \href{https://www.linkedin.com/in/tarannum03}{\bf tarannum03} \\
YouTube://  \href{https://www.youtube.com/channel/UCpdDBBU4c_2ycT-BDL8OtoA}{\bf Tarannum-youtube} \\
Hackerrank:// \href{https://www.hackerrank.com/tperween20}{\bf tperween20}\\

%%%%%%%%%%%%%%%%%%%%%%%%%%%%%%%%%%%%%%
%     COURSEWORK
%%%%%%%%%%%%%%%%%%%%%%%%%%%%%%%%%%%%%%

\section{Coursework}
Aerial Robotics\\
Robot Operating System with OpenCV\\
REDHAT Linux\\ 
MATLAB Onramp\\ 
Git and GitHub\\
Data Structure and Algorithm\\
Deep Learning\\
Python 3\\
\sectionsep


%%%%%%%%%%%%%%%%%%%%%%%%%%%%%%%%%%%%%%
%     SKILLS
%%%%%%%%%%%%%%%%%%%%%%%%%%%%%%%%%%%%%%

\section{Skills}
\location{Programming}
C  \textbullet{} Python \\
\location{Tools and Technology}
SOLIDWORKS \textbullet{} MATLAB \textbullet{} V-REP\\ \textbullet{} Linux \textbullet{} Git \textbullet{} GitHub \\
OpenCV \textbullet{} ROS \textbullet{} Deep Learning 
\sectionsep

\section{Societies & Clubs} 
\begin{tabular}{rll}
2019	     & Robotics Society NITH\\
2018	     & SPIC MACAY NITH\\
2018	     & NSS NITH\\
\end{tabular}
\sectionsep

%%%%%%%%%%%%%%%%%%%%%%%%%%%%%%%%%%%%%%
%
%     COLUMN TWO
%
%%%%%%%%%%%%%%%%%%%%%%%%%%%%%%%%%%%%%%

\end{minipage} 
\hfill
\begin{minipage}[t]{0.66\textwidth} 

%%%%%%%%%%%%%%%%%%%%%%%%%%%%%%%%%%%%%%
%     EXPERIENCE
%%%%%%%%%%%%%%%%%%%%%%%%%%%%%%%%%%%%%%

\section{Experience}
\runsubsection{Mysuru Counsulting group}
\descript{| Machine Learning}
\location{Feb 2021 - Present | India, Remote}
\sectionsep

\runsubsection{GirlScript Summer of Code}
\descript{|Open Source Contributor}
\location{Feb 2021 - Present | India, Remote}
\sectionsep

\runsubsection{The spark foundation}
\descript{| Computer vision }
\location{Jan 2021-Feb 2021 | India, Remote}
\vspace{\topsep} % Hacky fix for awkward extra vertical space
\begin{tightemize}
\item Implement an object detector which identifies the classes of the objects in an image or video.
\item Object Detection using SSD-MobileNetv3 and Implementation using Python, OpenCV and famous coco.names dataset. 
\end{tightemize}
\sectionsep


%%%%%%%%%%%%%%%%%%%%%%%%%%%%%%%%%%%%%%
%     RESEARCH
%%%%%%%%%%%%%%%%%%%%%%%%%%%%%%%%%%%%%%

\section{Projects}
\runsubsection{Alien Invasion Game}\\
This is a 2D game in which the aim is to shoot down a fleet of aliens as they drop down on the screen. I Made it using Python language, Pygame module in Linux (Ubuntu).  
\sectionsep

\runsubsection{Gesture controlled robotic arm}
\descript{| Leader Robotics Society NITH}
The 3D Model is a representation of a human arm. First, I modelled it using SOLIDWORKS and then I transferred it into Matlab / Simulink using the SimMechanics Link from Mat works.\\
Tools used were SOLIDWORKS, Matlab Simulation.
The code comes with the package (Arduino IO package) to enable the Simulink library. 
\sectionsep

\runsubsection{Obstacle Avoidance Robot with AI Integration}
\descript{| Robotics Society NITH}
It was an intelligent device that can automatically sense the obstacle in front of it and avoid them by turning itself in
another direction.
We used C++ language. I have done the hardware part and give training to the Bot.
\sectionsep

\runsubsection{Robot for Robowar}
\descript{| NIT Hamirpur}
The robot was a defensive bot.
My Team has got 3rd prize in the NIMBUS 2019 (Technical fest of NITH). 
\sectionsep


%%%%%%%%%%%%%%%%%%%%%%%%%%%%%%%%%%%%%%
%     AWARDS
%%%%%%%%%%%%%%%%%%%%%%%%%%%%%%%%%%%%%%

\section{Awards} 
\begin{tabular}{rll}
2020	     & top 37/3000+  & Summer of code with IIIT Una\\
2020	     & top 60/3000+   & Placemnt Maze coding Event by Coding Ninja\\
2019	     & 3rd prize      &  Robowar, conducted by the core team at NITH\\
2019     & Top 100 participant &  Olympiad 2.0 by National Engineering Olympiad  \\
\end{tabular}
\sectionsep


\end{minipage} 
\end{document}  \documentclass[]{article}
